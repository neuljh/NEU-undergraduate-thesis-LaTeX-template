\xiaosi \song
\noindent 附件1:
\begin{center}
    % \vspace*{16pt} % 空一行
    \hei\sihao{\appendixtitle}
\end{center}
\vspace*{14pt} % 空一行

毕业设计(论文)是实现毕业要求的基本单元,是支撑培养目标达成的主要判据。毕业论文撰写反映毕业论文工作的成效,是培养学生的工程(实践)意识、协作精神以及综合应用所学知识从事科学研究和解决实际问题能力的有效手段。掌握撰写毕业论文的基本能力是本科人才培养中的一个十分重要的环节。为了统一我校本科生毕业论文的撰写格式,特制定本规范。

本规范主要适用中文撰写的毕业论文。涉外专业用英文或其他外国语撰写的毕业论文可参照本规范执行。毕业论文由设计图纸和论文(说明书)两部分组成的,其图纸部分由各学院根据不同专业图纸的要求对图纸的版面尺寸大小、版式、数量、内容要求等制定详细的规范格式。

在遵照本规范的前提下,各学院(部)还可根据不同专业的特点对相关专业的毕业论文撰写格式提出更具体的要求。

一 \quad 内容要求

(一)论文题目

毕业设计(论文)选题应结合本专业的(工程)实际问题,论文题目应以最恰当、简明的词语准确概括整个论文的核心内容,避免使用不常见的缩略词、缩写字。中文题目一般不宜超过24个字,必要时可增加副标题。外文题目一般不宜超过12个实词。

(二)摘要和关键词

1.中文摘要和中文关键词

摘要内容应概括地反映出论文的主要内容,主要说明论文的研究目的、内容、方法、成果和结论。要突出论文的创新性成果,不要与引言相混淆。语言力求精练、准确。在摘要的下方另起一行,注明论文的关键词(3—5个)。

2.英文摘要和英文关键词

英文摘要内容与中文摘要相同。摘要后面注明英文关键词Keywords(3—5个)。

(三)目录

论文目录是论文的提纲,也是论文各章节组成部分的小标题。目录应按照章、节、条三级标题编写,采用阿拉伯数字分级编号,要求标题层次清晰。目录中的标题要与正文中的标题一致。

(四)正文

正文是毕业论文的主体和核心部分,不同学科专业和不同的选题可以有不同的写作方式。正文一般包括以下几个方面:

1.引言或背景

引言是论文正文的开端,引言应包括毕业论文选题的背景、目的和意义;对国内外研究现状和相关领域中已有的研究成果的简要评述;介绍本项研究工作研究设想、研究方法或实验设计、理论依据或实验基础;涉及范围和预期结果等。要求言简意赅,注意不要与摘要雷同或成为摘要的注解。

2.主体

论文主体是毕业论文的主要部分,必须言之成理,论据可靠,严格遵循本学科国际通行的学术规范。在写作上要注意结构合理、层次分明、重点突出,章节标题、公式图表符号必须规范统一。论文主体的内容根据不同学科有不同的特点,一般应包括以下几个方面:

(1)毕业设计(论文)总体方案或选题的论证;

(2)毕业设计(论文)各部分的设计实现,包括实验数据的获取、数据可行性及有效性的处理与分析、各部分的设计计算等;

(3)对研究内容及成果的客观阐述,包括理论依据、创新见解、创造性成果及其改进与实际应用价值等;

(4)论文主体的所有数据必须真实可靠,自然科学论文应推理正确、结论清晰;人文和社会学科的论文应把握论点正确、论证充分、论据可靠,恰当运用系统分析和比较研究的方法进行模型或方案设计,注重实证研究和案例分析,根据分析结果提出建议和改进措施等。

3.结论

结论是毕业论文的总结,是整篇论文的归宿。应精炼、准确、完整。着重阐述自己的创造性成果及其在本研究领域中的意义、作用,还可进一步提出需要讨论的问题和建议。

(五)中外文参考文献

毕业论文的撰写应本着严谨求实的科学态度,凡有引用他人成果之处,均应按论文中所引用的顺序列于文末,并且所有参考文献必须在正文中有引用标注。参考文献的著录均应符合国家有关标准(按照GB7714—2005《文后参考文献著录格式》执行)。一篇论著在论文中多处引用时,在参考文献中只应出现一次,序号以第一次出现的位置为准。

(六)相关的科研成果目录

包括本科期间发表的与学位论文相关的已发表论文或被鉴定的技术成果、发明专利等成果,应在成果目录中列出。此项不是必需项,空缺时可以略掉。

(七)致谢

表达作者对完成论文和学业提供帮助的老师、同学、领导、同事及亲属的感激之情。

(八)附录

对于一些不宜放在正文中的重要支撑材料,可编入毕业论文的附录中。包括某些重要的原始数据、详细数学推导、程序全文及其说明、复杂的图表、设计图纸等一系列需要补充提供的说明材料。

二 \quad 书写和打印规范

(一)书写及装订

论文按照本规范的要求单面或双面打印,论文裁切后规格为70g白色A4打印纸。一律左侧装订。封面用纸由学校统一制作,免费发放。

(二)字体和字号

论文题目 \hspace{6cm} 黑体二号

各章标题 \hspace{6cm} 黑体小二号

各节的一级标题 \hspace{5cm} 黑体四号

各节的二级标题 \hspace{5cm} 黑体小四号

各节的三级标题 \hspace{5cm}黑体小四号

款项 \hspace{7cm} 黑体小四号

正文 \hspace{7cm} 宋体小四号

中文摘要、结论、参考文献标题 \hspace{2cm}黑体小二号

中文摘要、结论、参考文献内容 \hspace{2cm} 宋体小四号

英文摘要标题 \hspace{5cm} Times New Roman大写加粗小二号

英文摘要内容 \hspace{5cm} Times New Roman小四号

中文关键词标题 \hspace{5cm} 黑体小四号

中文关键词 \hspace{6cm} 宋体小四号

英文关键词标题 \hspace{5cm} Times New Roman加粗小四号

英文关键词 \hspace{6cm} Times New Roman小四号

目录标题 \hspace{6cm} 黑体小二号

目录内容中章的标题 \hspace{4cm} 黑体四号

(含结论、参考文献、致谢、附录标题)

目录中其他内容 \hspace{5cm} 宋体小四号

论文页码 \hspace{2cm} 在页脚居中,用阿拉伯数字(Times New Roman 五号)连续编码

页眉与页脚 \hspace{6cm} 宋体五号

“东北大学本科生毕业设计(论文)”左排,各章标题居右排;页眉与正文之间用下划线分隔。

(三)封面

论文具体排版规范见封面示例,字体与字号要求如下:

学号 \hspace{7cm} (黑体五号)

密级 \hspace{7cm} (黑体五号)

东北大学本科生毕业设计(论文)\hspace{2cm} (宋体一号加粗居中)

论文题目 \hspace{7cm} (黑体二号居中)

学院名称 \hspace{7cm} (宋体小三号)

专业名称 \hspace{7cm} (宋体小三号)

学生姓名 \hspace{7cm} (宋体小三号)

指导教师 \hspace{7cm} (宋体小三号)

年\hspace{1cm}月 \hspace{7cm}  (宋体三号)

(四)学术声明

郑重声明\hspace{7cm}宋体加粗二号居中)

声明内容 \hspace{7cm}(宋体四号)

见学术声明示例。

(五)页面设置

页边距标准:页边距为上2.5cm,下2.5cm, 左3.0cm,右2.5cm,装订线0,页眉边距为1.5cm,页脚边距为1.5cm。
段前、段后及行间距:章标题的段前为0.8行,段后为0.5行;节标题段前为0.5行,段后0.5行;标题以外的文字行距为“固定值”23磅,字符间距为“标准”。

(六)摘要

摘要正文下空一行顶格打印“关键词”款项,每个关键词之间用“;”分开,最后一个关键词不打标点符号,英文摘要应另起一页。具体示例见中、英文摘要示例。.

(七)目录

目录应包括章、节、条三级标题,目录和正文中的标题题序统一按照“1……、1.1……、1.1.1……”的格式编写,目录中各章节题序中的阿拉伯数字用Times New Roman体。
目录的具体排版格式见目录示例。

(八)正文

正文各章节应拟标题,每章结束后应另起一页。标题要简明扼要,不应使用标点符号。各章、节、条的层次按照“1……、1.1……、1.1.1……”标识,条以下具体款项的层次依次按照“1.1.1.1”、“(1)”、“\textcircled{1}”标识。见正文示例。

(九)引文标示

引文标示应全文统一,采用方括号上标的形式置于所引内容最末句的右上角,引文编号用阿拉伯数字置于半角方括号中,用小四号字体,如:“……模式[3]”。各级标题不得使用引文标示。正文中如需对引文进行阐述时,引文序号应以逗号分隔并列排列于方括号中,如“文献[1,2,6-9]从不同角度阐述了……”

(十)名词术语

全文应统一科技名词术语、行业通用术语以及设备、元器件的名称。有国家标准的应采用标准中规定的术语,没有国家标准的应使用行业通用术语或名称。特定含义的名词术语或新名词应加以说明或注释。

(十一)物理量名称、符号与计量单位

论文中某一物理量的名称和符号应统一,一律采用国务院发布的《中华人民共和国法定计量单位》,单位名称和符号的书写方式,应采用国际通用符号。在不涉及具体数据表达时允许使用中文计量单位如“千克”。表达时刻应采用中文计量单位,如“下午3点10分”,不能写成“3h10min”。在表格中可以用“3:10PM”表示。
物理量变量符号用斜体,物理量常量符号用正体、计量单位符号均用正体。

(十二)数字

无特别约定情况下,一般均采用阿拉伯数字表示。年份一概用4位数字表示。小数的表示方法,一般情形下,小于1的数,需在小数点之前加0。
统计符号的字形格式,一般变量用斜体,常量用正体。

(十三)公式

公式应另起一行居中,统一用公式编辑器编辑。公式中字号不得大于正文字号,汉字字体为宋体,外文字母及符号采用Times New Roman体,常量用正体,变量用斜体表示。上、下标字母、数码和符号,位置高低要区别明显。公式与编号之间不加虚线。公式较长时应在“=”前转行或在“+、-、×、÷”运算符号处转行,若在“=”前转行,等号应在转行后的行首,若在“+、-、×、÷”运算符号处转行,运算符号转行后采用等号对齐的方式进行,公式的编号用圆括号括起来放在公式右边行末。
公式序号按章编排,如第3章第2个公式序号为“(3.2)”,附录中的第n个公式用序号“(An)”表示。文中引用公式时,采用“见公式(3.2)”表述。具体见公式图表示例。

(十四)表格

每一个表格都应有表标题和表序号。表序号一般按章编排,如第2章第4个表的序号为“表2.4”。表标题和表序之间应空一格,表标题中不能使用标点符号,表标题和表序号居中置于表上方(五号宋体,数字和字母为五号Times New Roman)。引用表格应在表标题的右上角加引文序号。
无特殊情况下,表与表标题、表序号为一个整体,不得拆开排版为两页。若一页无法显示,可采用在第二页添加“续表X.xx”方式进行。当页空白不够排版该表整体时,可将其后文字部分提前,将表移至次页最前面。
统计表一律采用三线表的标准格式,具体见公式图表示例。

(十五)图

插图应与文字内容相符,技术内容正确。所有制图应符合国家标准和专业标准。对无规定符号的图形应采用该行业的常用画法。
每幅插图应有图标题和图序号。图序号按章编排,如第1章第4幅插图序号为“图1.4”。图序号之后空一格写图标题,图序号和图标题居中置于图下方,用五号宋体,数字和字母为五号Times New Roman。引用图应在图标题右上角标注引文序号。图中若有分图,分图号用(a)、(b)等置于分图下、图标题之上。

图中的各部分中文或数字标示应置于图标题之上(有分图者置于分图序号之上),且字号不得大于正文字号。

图与图标题、图序号为一个整体,一个图标题下图片过多,若一页无法显示,可采用在第二页添加“图X.xx(续)”方式进行。当页空白不够排版该图整体时,可将其后文字部分提前,将图移至次页最前面。

对坐标轴必须用文字标识物理量名称,有数字标注的坐标图必须注明坐标单位。
具体见公式图表示例。

(十六)注释

注释是对论文中特定名词或新名词的注解。注释可用页末注或篇末注的一种。选择页末注的应在注释与正文之间加细线分隔,线宽度为1磅,线的长度不应超过纸张的三分之一宽度。同一页类列出多个注释的,应根据注释的先后顺序编排序号。字体为宋体五号,注释序号以“\textcircled{1}、\textcircled{2}”等数字形式标示在被注释词条的右上角。页末或篇末注释条目的序号应按照“\textcircled{1}、\textcircled{2}”等数字形式与被注释词条保持一致。

(十七)参考文献

参考文献的著录应符合国家标准,参考文献的序号左顶格,并用数字加方括号表示,与正文中的引文标示一致,如[1],[2]……。每一条参考文献著录均以 “.”结束。具体各类参考文献的编排格式如下(参考文献中的标点符号均以半角形式体现):

1.文献是期刊时,书写格式为:

[序号] 作者. 文章题目[J]. 期刊名, 出版年份, 卷号(期数): 起止页码.

(2)文献是图书时,书写格式为:

[序号] 作者. 书名[M]. 版次. 出版地: 出版单位, 出版年份: 起止页码.

3.文献是会议论文集时,书写格式为:

[序号] 作者. 文章题目[A]. 主编. 论文集名[C]. 出版地: 出版单位, 出版年份: 起止页码.

4.文献是学位论文时,书写格式为:

[序号] 作者. 论文题目[D]. 保存地: 保存单位, 年份.

5.文献是来自报告时,书写格式为:

[序号] 报告者. 报告题目[R]. 报告地: 报告会主办单位, 报告年份.

6.文献是来自专利时,书写格式为:

[序号] 专利所有者. 专利名称: 专利国别, 专利号[P]. 发布日期.

7.文献是来自国际、国家标准时,书写格式为: 

[序号] 标准代号. 标准名称[S]. 出版地: 出版单位, 出版年份.

8.文献来自报纸文章时,书写格式为: 
[序号] 作者. 文章题目[N]. 报纸名, 出版日期(版次).

9.文献来自电子文献时,书写格式为: 
[序号] 作者. 文献题目[电子文献及载体类型标识]. 电子文献的可获取地址, 发表或更新日期/引用日期(可以只选择一项).

电子参考文献建议标识:

[DB/OL]——联机网上数据库(database online)

[DB/MT]——磁带数据库(database on magnetic tape)

[M/CD] ——光盘图书(monograph on CD-ROM)

[CP/DK]——磁盘软件(computer program on disk)

[J/OL] ——网上期刊(serial online)

[EB/OL]——网上电子公告(electronic bulletin board online)

(十八)附录

论文附录依次用大写字母“附录A、附录B、附录C……”表示,附录内的分级序号可采用“附A1、附A1.1、附A1.1.1”等表示,图、表、公式均依此类推为“图A1、表A1、式(A1)”等。

(十九)印刷与装订顺序

毕业论文应按以下顺序装订:封面→学术声明→中文摘要→英文摘要→目录→正文→参考文献→附录→致谢
